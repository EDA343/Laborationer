\documentclass[a4paper,9pt,fleqn]{article}

\usepackage{mathtools}
\usepackage{enumerate}
\usepackage[retainorgcmds]{IEEEtrantools}
\usepackage{listings}
\usepackage{alltt}
\usepackage{pgf}
\usepackage{hyperref}
\setlength{\parindent}{0mm}
\setlength{\itemsep}{0ex}
\setlength{\parskip}{2ex}
\setlength{\parsep}{2ex}
\setlength{\partopsep}{0ex}
\setlength{\topsep}{0ex}

\newenvironment{outp}{\fontfamily{pcr}\selectfont}{\courier}

\lstset{
	aboveskip=5mm,
	belowskip=5mm,
	xleftmargin=5mm,
	language=,
	basicstyle=\scriptsize\outp,
	tabsize=2
}


\title{
{\bf Report}\\
\vspace{0.2cm}
%
Home Assignment, CCOM38\\
\vspace{1cm}
%
{\large Dennis Bennhage \& Hampus Lidin}\\
\vspace{10cm}
%
May 18th, 2015
}

\date{}

\begin{document}

\maketitle

\newpage

\begin{enumerate}[{Task} 1]
\item :
\begin{enumerate}[a)]
\item 
When running the {\outp ipconfig} command, you get all the
information about the different interfaces at the host computer. Running the
command in the Chalmers network, we get the following information
about the WLAN interface:
			
\begin{lstlisting}
> ifconfig

<output omitted>

Wireless LAN adapter Wireless Network Connection:

   Connection-specific DNS Suffix  . : eduroam.chalmers.se
   Link-local IPv6 Address . . . . . : fe80::6046:472a:63f5:468e%11
   IPv4 Address. . . . . . . . . . . : 129.16.187.2
   Subnet Mask . . . . . . . . . . . : 255.255.240.0
   Default Gateway . . . . . . . . . : 129.16.176.1

<output omitted>
\end{lstlisting}

The {\outp IPv4 Address} field shows information about the IPv4 configuration for that
interface. The unicast IP-address for the local network is then {\outp 129.16.187.2/20},
with the broadcast address {\outp 129.16.176.1/20}.

%However, the
%IP-address is within the private address space, so the router has to
%convert it into a globally unique IP-address using {\it Network Address
%Translation} (NAT). This address is provided to the router by the {\it Internet
%Service Provider} (ISP), and all traffic in the local network going out
%from the router will then use this address to communicate through the
%Internet.
    
\item 
The hostname can be found using the {\outp nslookup} command on the IP-address
obtained in the previous task:

\begin{lstlisting}
> nslookup 129.16.187.2

Server:		res1.chalmers.se
Address:	129.16.1.53

Name:		dhcp-187002.eduroam.chalmers.se
Address:	129.16.187.2
\end{lstlisting}

This tells us that the hostname for the IP-address {\outp 129.16.1.53} is
{\outp dhcp-187002.eduroam.chalmers.se}, and was given by the name server
{\outp res1.chalmers.se}.
\end{enumerate}

\item :
\begin{enumerate}[a)]
\item 
Answer.  
\\
\\
\item 
{\outp traceroute}, as the name suggests, traces the route of an IP-packet over
the Internet. It is implemented using the {\it User Datagram Protocol} (UDP),
to transport the packet from the host computer to a specified destination.
The reason UDP is used instead of the more reliable {\it Transmission Control
Protocol} (TCP) is that UDP maintains the route for all packets that it
sends, which is essential in order for {\outp traceroute} to work.
\\
\\
{\outp traceroute} initially sets the {\outp Time To Live} ({\outp TTL}) value
to 1 in the IP-header. Then after every three packets it sends, the {\outp TTL} value
increments by one. When the first router on the route then receives the first packet,
it will decrement the {\outp TTL} value by 1, resulting in a value of 0. This means that
the packet can not be forwarded any further and an {\it Internet Control Message Protocol}
(ICMP) message is generated. This message goes all the way back to the
source. In the mean time, {\outp traceroute} started counting the time from
when the packet was sent, and when the ICMP reply for that packet arrives, it stops the
timer. The same scenario occurs for the next two packets. When the ICMP reply for
the third packet arrives, {\outp traceroute} increments {\outp TTL} to 2. Then when the
next heap of three packets packet gets sent to the first router, the router decrements
{\outp TTL} to 1 and the packet is forwarded to the next router. That router then does the
same thing and generates an ICMP message (since {\outp TTL} now became 0) that goes back
to the host. This is repeated until the {\outp TTL} is large enough to reach the destination.
This way we can put together a table of three roundtrip times for each node on the route.
\end{enumerate}

\item :
\begin{enumerate}[a)]
\item 
The TCP/IP model has 4 layers, going from top to bottom; {\it application layer},
{\it transport layer}, {\it Internet layer} and {\it link layer}. When applications
in the application layer want to send data across the Internet, it has to make use of
several protocols in order for the data to successfully arrive 
destination. Each layer has its own protocols and responsibilities, that
are transparent to the other layers. For example, the Application layer can
use {\it HyperText Transfer Protocol} (HTTP) to send data, but how the data
is actually sent and how the other protocols work in the other layers, is
ignored by HTTP. When HTTP wants to send a message to a remote host, it
passes its data to the layer below, the transport layer. TCP is often used
to make a data transfer for HTTP. TCP breaks up the data into the transport {\it Protocol
Data Unit} (PDU) called {\it segment}, where every segment gets its own header.
This process is called segmentation.
\\
\\
TCP makes sure that the data is sent across a network in a reliable fashion.
How the segments will travel across the network (or any network), is up to the next
layer, the Internet layer. Here IP is used to address every IP-enabled end-point in
the network with a unique IP-address. IP encapsulates the segment into the
Internet PDU {\it packet}, also by adding a header with the essential
information about the data. How the packet is transferred between to
devices in the network is up to the link layer. The link layer's
responsibility lies in making sure that the data is sent across a link
between two devices. The packet gets a header and a trailer in an
encapsulation called {\it frame}. The frame can be sent on any medium, as long
as there is a protocol to handle the process. When a frame has been sent
across the medium, a new header and trailer is added for the next link
transfer.
\\
\\
When the first frame arrives at the destination, decapsulation begins.
The frame header and trailer is removed and the packet is revealed to the
Internet layer. After checking that the right packet has arrived, it
removes its header as well and sends it to the transport layer. The
transport layer also checks the validity of the segment, then decapsulate
it and passes it on to the right application. The application then checks
it message and use the data. 
\\
\\ 
\item 
A {\it Peer-to-Peer-Protocol} (P2PP) is a protocol that establishes communication
between two entities, or hosts, in the layer that it operates in. P2PPs are
only concerned about the responsibilities of their own layers, and do not bother
about the implementation of other layers and protocols. For example, TCP
establishes a connection over a network, by performing the {\it three-way handshake}
between two hosts. The two hosts operate in a similar way as each other, but none
of them care about how the data is being transferred across the network. They only
care about the reliability and verification of the transfer. That's why it's common
to say that PP2Ps operate between the layers of the peers.
\\
\\
\end{enumerate}

\item :
\begin{enumerate}[a)]
\item
Blablabla.

\begin{lstlisting}
www.chalmers.se:

	HEAD / HTTP/1.0

	HTTP/1.1 401 Unauthorized
	Content-Length: 16
	Content-Type: text/html; charset=utf-8
	Server: Microsoft-IIS/7.5
	SPRequestGuid: 1b34a246-0f4b-4010-9e90-6fd42daab1e4
	X-SharePointHealthScore: 0
	WWW-Authenticate: NTLM
	X-Powered-By: ASP.NET
	MicrosoftSharePointTeamServices: 14.0.0.7102
	X-MS-InvokeApp: 1; RequireReadOnly
	Date: Mon, 18 May 2015 08:20:33 GMT
	Connection: close
\end{lstlisting}	

The type of web server {\outp www.chalmers.se} is using is {\outp Microsoft-IIS/7.5}.
We get the response {\outp 401 Unauthorized}, which means that we do not have access
to URL resources without providing user authentication (log in somewhere with
a username and a password). The authentication protocol in this case, which
is specified in the {\outp WWW-Authenticate} field of the HTTP header, is {\outp NTLM}.

\begin{lstlisting}
	Content-Length is the size of the content that is being sent, in octets(bytes).
	Content-Type indicates what type of media is being sent.
	SPRequestGuid contains diagnostic information about server problems.
	X-SharePointHealthScore is a number between 0 and 10 where 0 indicates low server load and 10 indicates high server load.
	X-Powered-By specifies which technology is used to support the web application.
	MicrosoftSharePointTeamServices indicates which version of Microsoft SharePoint is installed.
	X-MS-InvokeApp specifies if the application wants to use DirectInvoke, which lets a user open for example a PDF without first saving it to their computer. 
\end{lstlisting}

RequiredReadOnly means that it opens in Read-Only mode.

\begin{lstlisting}
	www.tue.nl:

	HEAD / HTTP/1.0

	HTTP/1.1 301 Moved Permanently
	Date: Mon, 18 May 2015 08:24:17 GMT
	Server: Apache/2.2.22 (Ubuntu)
	X-Powered-By: PHP/5.3.10-1ubuntu3.18
	Location: http://www.tue.nl/
	Vary: Accept-Encoding
	Content-Type: text/html


	Connection closed by foreign host.


	{\outp www.tue.nl} is using the web server type {\outp Apache/2.2.22 (Ubuntu)}.
\end{lstlisting}

301 Move Permanently means that the resource we are requesting has been redirected
to a new URL. The new URL is specified in the Location field. In this case www.tue.nl
is redirecting to http://www.tue.nl/. The Vary field specifies which fields of the
request header to take into account when trying to find the right object in the cache.
\\
\\
\item 
Blablabla.

\begin{lstlisting}
www.chalmers.se:

	HEAD / HTTP/1.1

	HTTP/1.1 400 Bad Request
	Content-Length: 334
	Content-Type: text/html; charset=us-ascii
	Server: Microsoft-HTTPAPI/2.0
	Date: Mon, 18 May 2015 09:15:38 GMT
	Connection: close

	www.tue.nl:

	HEAD / HTTP/1.1

	HTTP/1.1 400 Bad Request
	Date: Mon, 18 May 2015 09:16:38 GMT
	Server: Apache/2.2.22 (Ubuntu)
	Vary: Accept-Encoding
	Connection: close
	Content-Type: text/html; charset=iso-8859-1
\end{lstlisting}
Blablabla.
\\
\\
\item 
Blablabla.

\begin{lstlisting}
www.chalmers.se:

	HEAD / HTTP/1.1
	Host: www.chalmers.se

	HTTP/1.1 302 Redirect
	Content-Length: 164
	Content-Type: text/html; charset=UTF-8
	Location: http://www.chalmers.se/Pages/default.aspx
	Server: Microsoft-IIS/7.5
	SPRequestGuid: f25d9db4-5345-4c46-b7fc-292c52258273
	X-SharePointHealthScore: 0
	X-Powered-By: ASP.NET
	MicrosoftSharePointTeamServices: 14.0.0.7102
	X-MS-InvokeApp: 1; RequireReadOnly
	Date: Mon, 18 May 2015 09:27:28 GMT

	www.tue.nl:

	HEAD / HTTP/1.1
	Host: www.tue.nl

	HTTP/1.1 200 OK
	Date: Mon, 18 May 2015 09:28:56 GMT
	Server: Apache/2.2.22 (Ubuntu)
	X-Powered-By: PHP/5.3.10-1ubuntu3.18
	Set-Cookie: fe_typo_user=e4577ddec4a01568b6eefae39b0dcf4b; path=/
	Expires: Mon, 18 May 2015 10:21:29 GMT
	Cache-Control: max-age=3144
	Vary: Accept-Encoding
	Content-Type: text/html; charset=utf-8

	Connection closed by foreign host.
\end{lstlisting}

In an HTTP 1.1 request you have to specify the name of the host of the resource
you are requesting. You can also specify a port number, but we did not do that
since it defaults to 80 if you do not specify one. If we do not include a host
name in the request, the server responds with 400 Bad Request, as we saw in 4b.
\\
\\
Say for example that we are hosting several different websites on a single machine.
They all share an IP address. To be able to send the request to the correct
website, we need to specify a host name. This is what the host header field is used
for; differentiating between multiple hosts on the same IP.

\end{enumerate}

\item :
\begin{enumerate}[a)]
\item 
{\it Cookie-files}, or {\it cookies}, are web files that are stored in the web browser.
When browsing a website, a pop-up window might appear asking the user if it wants to
enable cookies in the current web browser. If the user accepts, the website will keep
track of what the user do on the website and then tells the browser to store this
information as cookies on the computer.
\\
\\
When using the latest version of {\it Firefox} web browser, you can view what cookies
are stored on your computer. By going to the {\it Privacy} tab in {\it Options} and
clicking {\it remove individual cookies}, you can see all cookies in a list. As an example,
the following text shows the contents of a cookie after visiting Chalmers website:

\newpage

\begin{alltt}
\outp
\small
         Name:	__utmc
      Content:	192209784
       Domain:	.chalmers.se
         Path:	/
     Send For:	Any type of connection
      Expires:	At end of session
\end{alltt}

Here we see a cookie named {\outp \underline{\hspace{4mm}}utmc} that was stored while browsing
{\outp www.chalmers.se}. The cookie also contains information about the
domain, the path to the location of the file and when the cookie expires.
 
\item 
Answer.
\end{enumerate}

\item :
\begin{enumerate}[a)]
\item 
The {\outp nslookup} command can be used to find information about the DNS servers of a domain.
We will use {\outp utoronto.ca} (University of Toronto) for this task.  Here are the results of
our {\outp nslookup} for {\outp utoronto.ca}:

\begin{lstlisting}
> nslookup -type=mx utoronto.ca
	
Server:  res1.chalmers.se
Address:  129.16.1.53

Non-authoritative answer:
utoronto.ca     MX preference = 10, mail exchanger = k.mx.utoronto.ca
utoronto.ca     MX preference = 10, mail exchanger = d.mx.utoronto.ca
utoronto.ca     MX preference = 10, mail exchanger = b.mx.utoronto.ca
utoronto.ca     MX preference = 10, mail exchanger = c.mx.utoronto.ca
utoronto.ca     MX preference = 10, mail exchanger = g.mx.utoronto.ca
utoronto.ca     MX preference = 10, mail exchanger = a.mx.utoronto.ca
utoronto.ca     MX preference = 10, mail exchanger = l.mx.utoronto.ca
utoronto.ca     MX preference = 10, mail exchanger = e.mx.utoronto.ca
utoronto.ca     MX preference = 10, mail exchanger = f.mx.utoronto.ca
utoronto.ca     MX preference = 10, mail exchanger = j.mx.utoronto.ca

utoronto.ca     nameserver = bay.cs.utoronto.ca
utoronto.ca     nameserver = ns2.utoronto.ca
utoronto.ca     nameserver = ns1.utoronto.ca
utoronto.ca     nameserver = ns7.utoronto.ca
bay.cs.utoronto.ca      internet address = 128.100.1.1
ns1.utoronto.ca internet address = 128.100.100.129
ns2.utoronto.ca internet address = 128.100.72.168
ns7.utoronto.ca internet address = 162.243.71.42
\end{lstlisting}

To find the mail servers we set the type parameter to {\outp MX} (Mail eXchange).
We find that there are 10 mail servers for {\outp utoronto.ca}, with all of them
having equal priority (preference = 10). All of them having equal priority
means that we do not care which of the mail servers we use first. 
\\
\\
We also find the DNS servers and their respective IP addresses. 
\item 
We use {\outp www.google.com} for this task.

\begin{lstlisting}
> nslookup www.google.com

Server:  	res1.chalmers.se
Address:  	129.16.1.53

Non-authoritative answer:
Name:    	www.google.com
Addresses:  2a00:1450:4010:c02::68
         		74.125.205.106
         		74.125.205.103
         		74.125.205.105
         		74.125.205.99
         		74.125.205.147
         		74.125.205.104
\end{lstlisting}

We can see that www.google.com has 6 different IP addresses. Multiple IP addresses
can be used to balance load between web servers. There can be several copies of the
same web site, each having its own IP address but all using the same DNS servers. The
first user who sends a request is sent to the first IP address, the second person to
the second IP address and so on. This reduces the amount of requests to each web server
and provides redundancy in case a web server should go down.s
\\
\\
Multiple IP addresses can also be used to have different IP addresses for different services.
\end{enumerate}

\item :
\begin{enumerate}[a)]
\item 
TCP uses {\it timeout} to prevent eventual delays in the transfer. If one segment doesn't get
{\it acknowledged} {ACK'ed}, it would be bad if the whole transfer had to stall when waiting
for a reply. Instead it uses a timer to count down a segments maximal {\it Round Trip Time}
(RTT). When the timer counts to zero, the segment is considered lost and a duplicate segment
is sent. As the transfer proceeds, TCP will calculate new timeouts as it ``learns'' that 
the segments arrives either faster or slower. The tighter the timeout window is, the faster
TCP can discover that a segment is lost.   
\\
\\
\item 
When calculating the timeout window, TCP makes use of the {\it recorded} RTTs and the
{\it estimated} RTTs. We could just simply calculate the window by adding a little more
time to the recorded RTT, but that would lead to big variations and unstable transfer
flow. Instead we only count for a portion of the latest RTT sample and add it to the
complement portion of the previous calculation. This is what's called the estimated
RTT, and unlike the recorded RTT, it's less prone to fluctuation.
\\
\\
To actually determine a value for the timeout window, we need to calculate the deviation
of the sample RTT from the estimated RTT. Big fluctuations between these should affect
the deviation value more, and contrary small fluctuations should affect it less. By adding
an appropriate multiple of this deviation to the estimated RTT, we get an even calculation
of the timeout value, that adapts fairly quickly to data rate changes.  
\end{enumerate}

\end{enumerate}

\end{document}