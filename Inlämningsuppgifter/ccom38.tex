\documentclass[a4paper,9pt,fleqn]{article}

\usepackage{mathtools}
\usepackage{enumerate}
\usepackage[retainorgcmds]{IEEEtrantools}
\usepackage{listings}
\usepackage{pgf}
\usepackage{hyperref}
\setlength{\parindent}{0mm}
\setlength{\itemsep}{0ex}
\setlength{\parskip}{2ex}
\setlength{\parsep}{2ex}
\setlength{\partopsep}{0ex}
\setlength{\topsep}{0ex}

\newenvironment{outp}{\fontfamily{pcr}\selectfont}{\courier}

\lstset{
	aboveskip=5mm,
	belowskip=5mm,
	xleftmargin=5mm,
	language=bash,
	basicstyle=\scriptsize\outp,
	tabsize=2
}


\title{
{\bf Report}\\
\vspace{0.2cm}
%
Home Assignment, CCOM38\\
\vspace{1cm}
%
{\large Dennis Bennhage \& Hampus Lidin}\\
\vspace{10cm}
%
May 18th, 2015
}

\date{}

\begin{document}

\maketitle

\newpage

\begin{enumerate}[{Task} 1]
\item :
\begin{enumerate}[a)]
\item 
When running the {\outp ifconfig} command (in Mac OSX/Unix), you get all the
information about the different interfaces at the host computer. Running the
command in a home network, we get the following information
about the {\outp Ethernet0} (en0) interface:
			
\begin{lstlisting}
$ ifconfig

<output omitted>

en0: flags=8863<UP,BROADCAST,SMART,RUNNING,SIMPLEX,MULTICAST> mtu 1500
		 ether 20:c9:d0:7f:0f:25
		 inet6 fe80::22c9:d0ff:fe7f:f25\%en0 prefixlen 64 scopeid 0x4
		 inet 192.168.1.119 netmask 0xffffff00 broadcast 192.168.1.255
		 nd6 options=1<PERFORMNUD>
		 media: autoselect
		 status: active

<output omitted>
\end{lstlisting}

The {\outp inet} field shows information about the IPv4 configuration for that
interface. The unicast IP-address is then {\outp 129.16.182.106/20} and the
broadcast address for this network is {\outp 129.16.191.255/20}. However, the
IP-address is within the private address space, so the router has to
convert it into a globally unique IP-address using {\it Network Address
Translation} (NAT). This address is provided to the router by the {\it Internet
Service Provider} (ISP), and all traffic in the local network going out
from the router will then use this address to communicate through the
Internet.
    
\item 
The hostname can be found using the {\outp nslookup} command on the IP-address
obtained in the previous task:

\begin{lstlisting}
$ nslookup 192.168.1.1

Server:		192.168.1.1
Address:	192.168.1.1#53

119.1.168.192.in-addr.arpa	name = Hampus-MBP.huawei.com.
\end{lstlisting}

The answer is given by the name server {\outp 119.1.168.192.in-addr.arpa}. The
hostname that was given is {\outp Hampus-MBP.huawei.com}.
\end{enumerate}

\item :
\begin{enumerate}[a)]
\item 
Answer.  
\item 
Answer.
\end{enumerate}

\item :
\begin{enumerate}[a)]
\item 
Answer.  
\item 
Answer.
\end{enumerate}

\item :
\begin{enumerate}[a)]
\item 
Answer.  
\item 
Answer. 
\item 
Answer.
\end{enumerate}

\item :
\begin{enumerate}[a)]
\item 
Answer.  
\item 
Answer.
\end{enumerate}

\item :
\begin{enumerate}[a)]
\item 
Answer.  
\item 
Answer.
\end{enumerate}

\item :
\begin{enumerate}[a)]
\item 
Answer.  
\item 
Answer.
\end{enumerate}

\end{enumerate}

\end{document}